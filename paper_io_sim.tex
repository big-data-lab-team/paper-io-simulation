\documentclass[conference]{IEEEtran}

\begin{document}
\title{I/O simulation extension for SimGrid framework}
\author{Hoang-Dung Do}
\maketitle

\begin{abstract}
	\begin{itemize}
		\item The I/O bottleneck in HPC and the need of conducting HPC experiments.
		\item Difficulties in HPC experiments and simulation frameworks.
		\item Advantages of SimGrid, the missing of the ability to simulate page cache, the goal of this paper.
	\end{itemize}
\end{abstract}

	\section{Introduction}
		\begin{itemize}
			\item Role of high performance computing nowadays and the demand of HPC experiments.
			\item Difficulties in conducting high performance computing experiments and the need of simulation frameworks.
			\item Existing methods, simulators, simulation frameworks. 
			\item The advantages of SimGrid compared to others. The missing of the ability to simulate page cache in SimGrid.
			\item The objective of the paper: Add capability to simulate memory read/write, the impact of page cache on I/O in SimGrid.
		\end{itemize}
	\section{Related Work}			
		
		\subsection{Page cache}
			\begin{itemize}
				\item What is page cache? How it works. Effects and importance of page cache
				\item Introduce some existing strategies.
				\item Current implementation in Linux and some reasons for this to be implemented (implementation complexity, effectiveness, overhead, etc)
			\end{itemize}									

		\subsection{Simulators}
		Discuss some ecisting methods, simulation frameworks to conduct HPC experiments. Compare pros and cons (accuracy, simulation time, usability) of some simulators (SimGrid, GridSim) .
			
	\section{Method}

		\subsection{Principle of the simulator}

			\begin{itemize}
				\item Approach: generalize dirty data, dirty ratio, cache eviction strategy implemented in Linux. 
				\item Some implemented details of the simulator in SimGrid.
			\end{itemize}

		\subsection{Implementation}
			\begin{itemize}
				\item Which features of memory are implemented.
				\item Level of granularity, how features are implemented.
				\item Specific implementation detail in python and SimGrid.
			\end{itemize}

		\subsection{Experiments}
			Describe data, workflow, number of tasks, task details, environment of each experiment.
	
			\subsubsection{Expriment 1}
				A single pipeline running one node.
			\subsubsection{Expriment 2}
				Multiple pipelines running in parallel on multiple nodes.
			\subsubsection{Expriment 3}
				Same as Experiment 2 but nodes write to a shared file system.
			\subsubsection{Expriment 4}
				A real pipeline (for example a pipeline with nighres)

	\section{Results}
	
		\begin{itemize}

			\item Quantized results: 
				\begin{itemize}
					\item Errors of simulation time and memory used compared to real results.
					\item Simulation time compared to baseline SimGrid.
				\end{itemize} 

			\item Ability of the model to generalize trends seen in memory (amount of dirty data, page cache) and disk throughput.

		\end{itemize}

	\section{Discussion and Future Work}

\end{document}